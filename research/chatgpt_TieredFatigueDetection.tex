\documentclass[11pt]{article}

\usepackage[margin=1in]{geometry}
\usepackage{amsmath,amssymb}
\usepackage{bm}
\usepackage{hyperref}

\title{Tiered Fatigue Detection and Orthogonal Recapitulation Model}
\author{}
\date{}

\begin{document}

\maketitle

\section{System Overview}

We formalize crystallization detection as a multi-component dynamical system operating over embedding space and probability space.

Let:
\[
e_t \in \mathbb{R}^d
\]
be the embedding vector of the model output at turn $t$.

Let:
\[
p_t \in \mathbb{R}^V
\]
be the softmax probability distribution over vocabulary size $V$.

\section{Composite Fatigue Score}

We define a weighted fatigue score:

\begin{equation}
F_t = \alpha S_t + \beta E_t + \gamma N_t
\end{equation}

with weights:

\[
\alpha = 0.4, \quad \beta = 0.3, \quad \gamma = 0.3
\]

Each component measures a different crystallization pressure.

\section{Similarity Component ($S_t$)}

Rolling repetition across the last $k$ turns:

\begin{equation}
S_t = \frac{1}{k} \sum_{i=1}^{k}
\cos(e_t, e_{t-i})
\end{equation}

High $S_t$ indicates convergence toward a semantic attractor basin.

\section{Entropy Collapse Component ($E_t$)}

Shannon entropy of token probabilities:

\begin{equation}
H_t = -\sum_{i=1}^{V} p_{t,i} \log p_{t,i}
\end{equation}

Normalize by maximum entropy:

\begin{equation}
\tilde{H}_t = \frac{H_t}{\log V}
\end{equation}

Define entropy collapse:

\begin{equation}
E_t = 1 - \tilde{H}_t
\end{equation}

High $E_t$ indicates confidence concentration (reduced diversity).

\section{Novelty Drift Component ($N_t$)}

Define historical centroid:

\begin{equation}
\bar{e}_t = \frac{1}{k} \sum_{i=1}^{k} e_{t-i}
\end{equation}

Define drift:

\begin{equation}
D_t = 1 - \cos(e_t, \bar{e}_t)
\end{equation}

Define stagnation:

\begin{equation}
N_t = 1 - D_t
\end{equation}

High $N_t$ indicates trajectory stagnation.

\section{Tiered Thresholding}

Two operational thresholds:

\begin{align}
\theta_1 &= 0.68 \quad \text{(soft disclosure)} \\
\theta_2 &= 0.84 \quad \text{(hard recapitulation trigger)}
\end{align}

Decision rules:

\[
F_t > \theta_1 \Rightarrow \text{process disclosure}
\]

\[
F_t > \theta_2 \Rightarrow \text{structural recapitulation}
\]

\section{Orthogonal Recapitulation Perturbation}

When $F_t > \theta_2$, compute structured escape.

Define historical subspace:

\[
H = \text{span}\{ e_{t-1}, e_{t-2}, \dots, e_{t-k} \}
\]

Project current embedding:

\begin{equation}
\text{proj}_H(e_t)
\end{equation}

Define orthogonal contrast:

\begin{equation}
v_t = e_t - \text{proj}_H(e_t)
\end{equation}

Normalize:

\begin{equation}
\hat{v}_t = \frac{v_t}{\|v_t\|}
\end{equation}

Perturbed embedding:

\begin{equation}
e'_t = e_t + \lambda \hat{v}_t
\end{equation}

where
\[
0.05 \le \lambda \le 0.15
\]

This enforces basin escape while preserving semantic continuity.

\section{Interpretation}

\begin{itemize}
\item $S_t$ detects repetition.
\item $E_t$ detects over-confidence collapse.
\item $N_t$ detects trajectory stagnation.
\item $F_t$ measures composite crystallization pressure.
\item Orthogonal perturbation enforces controlled basin transition.
\end{itemize}

The model is locally computable when embeddings and logits are available and does not require modification of transformer weights.

\end{document}

